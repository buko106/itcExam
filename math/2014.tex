\documentclass[a4j]{jarticle}
%%  packages
\usepackage{amsmath,amssymb,ascmac}
\usepackage{bm}
\usepackage[dvipdfmx]{graphicx}
\usepackage{listings}
\usepackage[english]{babel}
\lstset{
 	%枠外に行った時の自動改行
 	breaklines = true,
 	%標準の書体
        basicstyle=\ttfamily\footnotesize,
        commentstyle=\footnotesize\bfseries,
        keywordstyle=\footnotesize\bfseries,
 	%枠 "t"は上に線を記載, "T"は上に二重線を記載
	%他オプション:leftline,topline,bottomline,lines,single,shadowbox
 	frame = single,
 	%frameまでの間隔(行番号とプログラムの間)
 	framesep = 5pt,
 	%行番号の位置
 	numbers = left,
	%行番号の間隔
 	stepnumber = 1,
	%タブの大きさ
 	tabsize = 4,
 	%キャプションの場所("tb"ならば上下両方に記載)
 	captionpos = t
}

%% math commands
\let \ds \displaystyle
\newcommand{\idiff}[3]{
  \frac{d^{#1} #2}{d #3^{#1}}
}
\newcommand{\diff}[3]{
  \frac{\mathrm{d}^{#1} #2}{\mathrm{d} #3^{#1}}
}
\newcommand{\pdiff}[3]{
  \frac{\partial^{#1} #2}{\partial #3^{#1}}
}



%% title configuration
\title{東京大学大学院情報理工学系研究科2014年度過去問}
\author{}
\date{}


%% headings
\pagestyle{headings}
\markright{東京大学大学院情報理工学系研究科2014年度過去問}




\begin{document}
%%  begin title page
\thispagestyle{empty}
\maketitle
\pagebreak

\section{}

\begin{screen}
 実正方行列$M$は,$M=M^T$を満たすとき対称行列という.ただし,$M^T$は$M$の転置行列を表す.以下の設問に答えよ.
\end{screen}

\begin{screen}
 (1)以下の対称行列$A$のすべての固有値と,それらの固有に対応する固有ベクトルを求めよ.
 $$A=\begin{pmatrix}
    1 & -1 & 0 \\
    -1 & 1 & -1 \\
    0 & -1 & 1
   \end{pmatrix}$$
\end{screen}


\begin{screen}
 (2)実正方行列$M$が対称行列ならば、その固有値はすべて実数となることを証明せよ.
\end{screen}


\begin{screen}
 (3)実正方行列$M$の固有値がすべて実数であっても$M$は必ずしも対称行列であるとは限らない.そのような行列$M$の具体例を1つ挙げよ.
\end{screen}


\begin{screen}
 (4)$\bm{u}=\begin{pmatrix}x \\ y \\ z\end{pmatrix}$を原点を除く3次元実ベクトルとする.ここで,設問(1)の対称行列$A$を用いて,関数$f(x,y,z)$を次のように定義する.
 $$f(x,y,z)=\frac{\bm{u}^TA\bm{u}}{\bm{u}^T\bm{u}}$$
 ただし,$\bm{u}^T$は$\bm{u}$の転置を表す.また,関数$g(x,y,z)$を次のように定義する.
 $$g(x,y,z)=\frac{xy+yz}{x^2+y^2+z^2}$$
 このとき,次の関係式が成り立つことを示せ.
 $$f(x,y,z)=1-2g(x,y,z)$$
\end{screen}


\begin{screen}
 (5)設問(4)の関数$g(x,y,z)$について,次の不等式が成立つことを設問(1)の対称行列$A$の固有値分解を用いて示せ.
 $$-\frac{1}{\sqrt{2}}\leq g(x,y,z) \leq \frac{1}{\sqrt{2}}$$
\end{screen}

\section{}

\begin{screen}
 区間$[-1,1]$で定義された実関数$f(x),g(x)$に対して,
 $$(f,g)=\int_{-1}^1f(x)g(x)\mathrm{d}x$$
 とおく.以下の設問に答えよ.
\end{screen}


\begin{screen}
 (1)以下で定義される多項式$q_0(x),q_1(x),q_2(x)$に対して$(q_0,q_1),$ $(q_0,q_2),$ $(q_1,q_2),$ $(q_0,q_0),$ $(q_1,q_1),$ $(q_2,q_2)$を計算せよ.
 $$q_0(x)=1,\quad q_1(x)=x,\quad q_2(x)=3x^2-1$$
 ただし,任意の奇関数$h(x)$に対して$\ds \int_{-1}^1h(x)\mathrm{d}x = 0$であることを用いてよい.
\end{screen}

\begin{align*}
 (q_0,q_1) &= \int_{-1}^1 x \mathrm{d}x = 0 \\
 (q_0,q_2) &= \int_{-1}^1 (3x^2 -1)\mathrm{d}x = 0 \\
 (q_1,q_2) &= \int_{-1}^1 (3x^3 -x)\mathrm{d}x =0 \\
 (q_0,q_0) &= \int_{-1}^1 \mathrm{d}x = 2\\
 (q_1,q_1) &= \int_{-1}^1 x^2\mathrm{d}x =\frac{2}{3}\\
 (q_2,q_2) &= \int_{-1}^1 (9x^4 - 6 x^2 + 1)\mathrm{d}x = 2 \left[\frac{9}{5}x^5-2x^3 + x\right]_0^1 = \frac{8}{5}\\
\end{align*}

\begin{screen}
 (2) $p_k(x)$ を $x$ の $k$ 次多項式(ただし,$x^k$ の項の係数は $0$ でないとする)とし,多項式関数の列 $p_0(x),p_1(x),\cdots$ を考える.以下では,$N$ 個の関数の組 $\left\{p_0(x),p_1(x),\cdots,p_{N-1}(x)\right\}$ が $[0,N-1]$ 内の任意の整数 $i,j$ に対して
 $$(p_i,p_j)=\begin{cases}
              1 &(i=j)\\
              0 &(i \neq j)
             \end{cases}$$
 を満たすとき,その組は正規直交条件を満たすという.
 \begin{itemize}
  \item[(2-1)] 設問(1)で定義した$q_0(x),q_1(x),q_2(x)$を用いて,正規直交条件を満たす関数の組$\left\{p_0(x),p_1(x),p_2(x)\right\}$を一つ求めよ.
  \item[(2-2)] 関数の組$\left\{p_0(x),p_1(x),p_2(x),p_3(x)\right\}$が正規直交条件を満たすような$p_3(x)$を一つ求めよ.ただし,$p_0(x),p_1(x),p_2(x)$は前問(2-1)で求めたものとする.
 \end{itemize}
\end{screen}

\begin{itemize}
 \item[(2-1)] $q_0(x),q_1(x),q_2(x)$は既に直交しているので,正規化することで$p_0(x),p_1(x),p_2(x)$を得る.
              \begin{align*}
               p_0(x) &= \frac{q_0(x)}{\sqrt{(q_0,q_0)}} = \frac{1}{\sqrt{2}} \\
               p_1(x) &= \frac{q_1(x)}{\sqrt{(q_1,q_1)}} = \sqrt{\frac{3}{2}} x \\
               p_2(x) &= \frac{q_2(x)}{\sqrt{(q_2,q_2)}} = \sqrt{\frac{8}{5}} (3x^2-1)
              \end{align*}
 \item[(2-2)] $q_3(x)=x^3$と定めて,グラムシュミットの直交化を用いる.
              \begin{align*}
               u_3(x) &= q_3(x) - \frac{(p_2,q_3)}{(p_2,p_2)}p_2(x) - \frac{(p_1,q_3)}{(p_1,p_1)}p_1(x) - \frac{(p_0,q_3)}{(p_0,p_0)}p_0(x) \\
               &= x^3 - \frac{3}{5} x \\
               p_3(x) &= \frac{u_3(x)}{\sqrt{(u_3,u_3)}} \\
               &= \sqrt{\frac{7}{8}}(5x^3 - 3x)
              \end{align*}
\end{itemize}

\begin{screen}
 (3) いま,関数の組$\left\{p_0(x),p_1(x),\cdots,p_{N-1}(x)\right\}$が正規直交条件を満たしているとする.このとき小問(2-2)と同様に,関数の組$\left\{p_0(x),p_1(x),\cdots,p_{N}(x)\right\}$が正規直交条件をみたすように$p_N(x)$を定めることができる.次の手順により,この$p_N(x)$は符号を除いて一意でることを示せ.
 \begin{itemize}
  \item[(3-1)] 一般に任意の$N$次多項式$f_N(x)$は
               $$f_N(x)=\sum_{k=0}^Nc_kp_k(x)$$
               と書ける.この係数 $c_k$ $(k=0,\cdots,N)$を$p_0(x),p_1(x),\cdots,p_{N}(x)$と$f_N(x)$を用いて表せ.
  \item[(3-2)] $p_N(x)$とは異なる$N$次多項式関数$\tilde{p}_N(x)$が存在して,関数の組$\left\{p_0(x),p_1(x),\cdots,\right.$ $\left.p_{N-1}(x),\tilde{p}_{N}(x)\right\}$も正規直交条件を満たしたとする.このとき,前問(3-1)で$f_N(x)=\tilde{p}_N(x)$の場合を考えることで$\tilde{p}_N(x)=-p_N(x)$を示せ.
 \end{itemize}
\end{screen}

\begin{itemize}
 \item[(3-1)] 正規直交条件より,$\ds (f_N,p_k) = \sum_{i=0}^N c_i (p_i,p_k) = c_k$
 \item[(3-2)] $0 \leq k < N$のとき,$c_k = (\tilde{p}_N,p_k) = 0$ であるので,$\tilde{p}_N(x)=c_N p_N(x)$となる.\\ $(\tilde{p}_N,\tilde{p}_N)=(c_kp_N,c_kp_N)=c_k^2=1$ \\ $\therefore c_k = \pm 1$ \quad よって$p_N$と$\tilde{p}_N$が異なるとき,$\tilde{p}_N(x)=-p_N(x)$

               
\end{itemize}

\section{}

\begin{screen}
 独立な確率変数の列$x_0,x_1,x_2,\cdots$において,各$x_i$ $(i=0,1,2,\cdots)$は,確率$p$で$1$の値をとり,確率$1-p$で$0$の値をとるものとする.以下の設問に答えよ.
\end{screen}


\begin{screen}
 (1)確率変数列$x_0,x_1,x_2,\cdots$に関して,以下の小問に答えよ.
 \begin{itemize}
  \item[(1-1)]$x_i$の分散を求めよ.また,$x_0=x_1=1$となる確率を求めよ.
  \item[(1-2)]$k(k\geq 0)$を$x_k=x_{k+1}$が成り立つ最小の整数とする.たとえば,$x_0,x_1,x_2,\cdots$が$1,0,0,1,0,1,1,1,0,0,\cdots$ならば$k=1$となり$x_k=0$となる.\\
              $x_k=1$となる確率を求めよ.
 \end{itemize}
\end{screen}

\begin{itemize}
 \item[(1-1)] $x_i$の期待値は$0\cdot (1-p) + 1\cdot p = p$である.よって分散は$(0-p)^2 (1-p) + (1-p)^2 p = p(1-p)$
 \item[(1-2)] (i)$x_0=1$のとき.$1,0$を$n$回繰り返した後に$1,1$となる.従ってその確率は$\ds \sum_{n=0}^\infty p^n(1-p)^n\cdot p^2$ \\
              (ii)$x_0=0$のとき,$0$の後に$1,0$を$n$回繰り返した後に$1,1$となる.従ってその確率は \\ $\ds \sum_{n=0}^\infty (1-p) \cdot p^n(1-p)^n\cdot p^2$  \\
              (i)+(ii)によって,確率は
              $$p^2(2-p)\sum_{n=0}^\infty p^n(1-p)^n = \frac{p^2(2-p)}{1-p(1-p)}$$
\end{itemize}

\begin{screen}
 (2)確率変数列$x_0,x_1,x_2,\cdots$をもとに確率変数列$y_0,y_1,y_2,\cdots$を以下のように定める.
 \begin{align*}
  y_0 &= 1 & \\
  y_{i+1} &= y_i + \alpha(x_i-y_i) & (i=0,1,2,\cdots)
 \end{align*}
 ただし,$0<\alpha<1$と仮定する.以下の小問に答えよ.
 \begin{itemize}
  \item[(2-1)] $\ds y_n=(1-\alpha)^n + \sum_{i=0}^{n-1}(1-\alpha)^{n-i-1}\alpha x_i$ $(n=1,2,\cdots)$であることを示せ.
  \item[(2-2)] $y_n$の期待値$E_n$と分散$V_n$を求めよ.
  \item[(2-3)] $\ds E_\infty = \lim_{n\rightarrow \infty}E_n,V_\infty = \lim_{n\rightarrow \infty}V_n$とおく.$\ds \frac{1}{2}<p<\frac{3}{4}$であるとき,
               $$E_\infty - \sqrt{V_\infty} \geq \frac{1}{2}$$
               を満たす$\alpha$の最大値を求めよ.
 \end{itemize}
\end{screen}

\begin{itemize}
 \item[(2-1)] 数学的帰納法により示す.(i) $n=1$のとき.$y_1 = (1-\alpha) + \alpha x_0 = y_0 + \alpha(x_0-y_0)$で成立する.(ii) $n=k(k\geq 1)$で成り立つことを仮定する.
              $\ds y_{k+1}=(1-\alpha)y_k + \alpha x_k = (1-\alpha)^{k+1} + \sum_{i=0}^{k}(1-\alpha)^{k-i}\alpha x_i $となり$n=k+1$のときも成立する.
 \item[(2-2)] $\ds E_n = E[(1-\alpha)^n] + \sum_{i=0}^{n-1}(1-\alpha)^{n-i-1}\alpha E[x_i] = (1-\alpha)^n + p\alpha \sum_{i=0}^{n-1}(1-\alpha)^i = p + (1-p)(1-\alpha)^n$ \\
              $x_i$の独立性より \\
              \begin{align*}
               V_n &= V[(1-\alpha)^n] + \sum_{i=0}^{n-1}(1-\alpha)^{2(n-i-1)}\alpha^2 V[x_i] \\
               &= 0 + p(1-p)\alpha^2 \sum_{i=0}^{n-1}(1-\alpha)^{2i} \\
               &= p(1-p)\alpha^2\frac{1-(1-\alpha)^{2n}}{1-(1-\alpha)^2} \\
               &= p(1-p)\alpha \frac{1-(1-\alpha)^{2n}}{2-\alpha}
              \end{align*}
 \item[(2-3)] $\ds E_\infty=p,V_\infty = p(1-p)\frac{\alpha}{2-\alpha}$となる.$\ds p>\frac{1}{2}$より,
              \begin{align*}
               &E_\infty - \sqrt{V_\infty} \geq \frac{1}{2} \\
               &\Leftrightarrow \left(E_\infty - \frac{1}{2} \right)^2  \geq V_\infty \\
               &\Leftrightarrow p^2 - p + \frac{1}{4} \geq p(1-p)\frac{\alpha}{2-\alpha} \\
               &\Leftrightarrow \alpha \leq 2\left(p-\frac{1}{2}\right)^2
              \end{align*}
              となり,最大値は$\ds \alpha = 2\left(p-\frac{1}{2}\right)^2$
\end{itemize}

\end{document}
