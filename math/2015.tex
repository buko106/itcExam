\documentclass[a4j]{jarticle}
%%  packages
\usepackage{amsmath,amssymb,ascmac}
\usepackage{bm}
\usepackage[dvipdfmx]{graphicx}
\usepackage{listings}
\usepackage[english]{babel}
\lstset{
 	%枠外に行った時の自動改行
 	breaklines = true,
 	%標準の書体
        basicstyle=\ttfamily\footnotesize,
        commentstyle=\footnotesize\bfseries,
        keywordstyle=\footnotesize\bfseries,
 	%枠 "t"は上に線を記載, "T"は上に二重線を記載
	%他オプション:leftline,topline,bottomline,lines,single,shadowbox
 	frame = single,
 	%frameまでの間隔(行番号とプログラムの間)
 	framesep = 5pt,
 	%行番号の位置
 	numbers = left,
	%行番号の間隔
 	stepnumber = 1,
	%タブの大きさ
 	tabsize = 4,
 	%キャプションの場所("tb"ならば上下両方に記載)
 	captionpos = t
}

%% math commands
\let \ds \displaystyle
\newcommand{\idiff}[3]{
  \frac{d^{#1} #2}{d #3^{#1}}
}
\newcommand{\diff}[3]{
  \frac{\mathrm{d}^{#1} #2}{\mathrm{d} #3^{#1}}
}
\newcommand{\pdiff}[3]{
  \frac{\partial^{#1} #2}{\partial #3^{#1}}
}



%% title configuration
\title{東京大学大学院情報理工学系研究科2015年度過去問}
\author{}
\date{}


%% headings
\pagestyle{headings}
\markright{東京大学大学院情報理工学系研究科2015年度過去問}




\begin{document}
%%  begin title page
\thispagestyle{empty}
\maketitle
\pagebreak

\section{}


\section{}


\section{}

\begin{screen}
 以下の設問に答えよ.
\end{screen}

\begin{screen}
 (1) $X$ を実数値をとる確率変数とし,$t$を実変数として,$\phi_X(t)$を
 $$\phi_X(t) = E_X\left[e^{tX}\right]$$
 として定める.ここに,$E_X\left[\cdot\right]$は$X$に関する期待値を表す.$\phi_X(t)$が$t=0$の近傍で有限であるとき,$X$の平均と分散を$\phi'_X(0)$と$\phi''_X(0)$を用いて表せ.ここに,$\phi'_X(t)$,$\phi''_X(t)$はそれぞれ$\phi_X(t)$の$t$に関する1階微分,2階微分を表す.
\end{screen}

\begin{screen}
 (2) 互いに独立な確率変数列 $X_1,X_2,\cdots,X_N$ に対して,各 $X_j$ が平均 $\mu$,分散 $\sigma^2$ の同一の1次元正規分布に従うとする.つまり,各$X_j$の従う確率密度関数が,
 $$p(X_j=x) = \frac{1}{\sqrt{2 \pi}\sigma}\exp\left(- \frac{(x-\mu)^2}{2\sigma^2}\right)$$
 で与えられるとする.このとき,$\phi_{X_j}(t)$ を求めよ.また,
 $$ Y = X_1 + X_2 + \cdots + X_N$$
 の従う確率分布を求めよ.ただし,一般に,2つの確率変数$Z$と$W$に対して,$t$の関数として,$\phi_Z(t)=\phi_W(t)$であれば,$Z$と$W$が従う確率分布は等しいという事実を用いてもよい.
\end{screen}
\end{document}
