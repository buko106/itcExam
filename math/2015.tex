\documentclass[a4j]{jarticle}
%%  packages
\usepackage{amsmath,amssymb,ascmac}
\usepackage{bm}
\usepackage[dvipdfmx]{graphicx}
\usepackage{listings}
\usepackage[english]{babel}
\lstset{
 	%枠外に行った時の自動改行
 	breaklines = true,
 	%標準の書体
        basicstyle=\ttfamily\footnotesize,
        commentstyle=\footnotesize\bfseries,
        keywordstyle=\footnotesize\bfseries,
 	%枠 "t"は上に線を記載, "T"は上に二重線を記載
	%他オプション:leftline,topline,bottomline,lines,single,shadowbox
 	frame = single,
 	%frameまでの間隔(行番号とプログラムの間)
 	framesep = 5pt,
 	%行番号の位置
 	numbers = left,
	%行番号の間隔
 	stepnumber = 1,
	%タブの大きさ
 	tabsize = 4,
 	%キャプションの場所("tb"ならば上下両方に記載)
 	captionpos = t
}

%% math commands
\let \ds \displaystyle
\newcommand{\idiff}[3]{
  \frac{d^{#1} #2}{d #3^{#1}}
}
\newcommand{\diff}[3]{
  \frac{\mathrm{d}^{#1} #2}{\mathrm{d} #3^{#1}}
}
\newcommand{\pdiff}[3]{
  \frac{\partial^{#1} #2}{\partial #3^{#1}}
}



%% title configuration
\title{東京大学大学院情報理工学系研究科2015年度過去問}
\author{}
\date{}


%% headings
\pagestyle{headings}
\markright{東京大学大学院情報理工学系研究科2015年度過去問}




\begin{document}
%%  begin title page
\thispagestyle{empty}
\maketitle
\pagebreak

\section{}

\begin{screen}
 $A,b$を以下のように定義する.
 
 $\ds A=\left(
 \begin{array}{rrr}
      -3 & 0 & 0 \\
      -2 & -3 & 1 \\
      2 & -3 & -3
 \end{array}\right),
 b=\left(\begin{array}{c}1 \\ 1 \\ 0\end{array}\right)$

 また,$x=(x_1 \hspace{1mm} x_2 \hspace{1mm} x_3)^T$のスカラー値関数$f(x)$の$x$に関する偏微分$\ds \pdiff{}{}{x}f(x)$を
 $$\pdiff{}{}{x}f(x) = \left(\pdiff{}{}{x_1}f(x)\hspace{2mm}\pdiff{}{}{x_2}f(x)\hspace{2mm}\pdiff{}{}{x_3}f(x)\right)$$
 と定義し,$f(x)$の停留点を$\ds \pdiff{}{}{x}f(x)=(0 \hspace{1mm} 0 \hspace{1mm} 0)$となる$x$と定義する.なお,$x^T$は$x$の転置を表すものとする.以下の設問に答えよ.
\end{screen}

\begin{screen}
 (1) 行列$A$の特性方程式を記せ.
\end{screen}

\begin{align*}
 \ds \Phi_A(\lambda) &= |\lambda I - A| \\
 &=
 \begin{vmatrix}
 \lambda + 3 & 0 & 0 \\
 2 & \lambda +3 & -1 \\
 -2 & 3 & \lambda + 3
 \end{vmatrix} \\&= (\lambda+3)
 \begin{vmatrix}
 \lambda + 3 & -1 \\
 3 & \lambda+3
 \end{vmatrix} \\&= \lambda^3+9\lambda^2+30\lambda+36
\end{align*}

よって,特性方程式は$\ds \Phi_A(\lambda) = \lambda^3+9\lambda^2+30\lambda+36 = 0$.

\begin{screen}
 (2) 行列$C$は行列$A$と単位行列$I$を用いて,$C=A^5 + 9A^4 + 30 A^3 + 36 A^2 + 30 A + 9 I $と表される. 行列$C$を求めよ.
\end{screen}

$\ds \Phi_A(A) = A^3 + 9A^2 + 30 A + 36 I= O$となるので,$\ds C= A^2 \Phi_A(A) + 30A + 9I =
\begin{pmatrix}
 -81 & 0 & 0 \\
 -60 & -81 & 30 \\
 60 & -90 & -81
\end{pmatrix}$

\begin{screen}
 (3) $x^TAx$を$x$について偏微分せよ.
\end{screen}

$x^TAx = -3x_1^2-3x_2^2-3x_3^2-2x_1x_2-2x_2x_3+2x_3x_1$である.よって,
$$\pdiff{}{}{x}\left(x^TAx\right) = \bigl( -6x_1-2x_2+2x_3 \quad -2x_1-6x_2-2x_3 \quad 2x_1-2x_2-6x_3 \bigr)$$

(別解) $\ds\pdiff{}{}{x}\left(x^TAx\right) = (A+A^T)x$ を利用する.

\begin{screen}
 (4) 任意のベクトル$x$に対して,$x^TAx=x^T\tilde{A}x$が成り立つ対称行列$\tilde{A}$を1つ求めよ.また,行列$\tilde{A}$の固有値$\lambda_1,\lambda_2,\lambda_3$ $(\lambda_1\geq\lambda_2\geq\lambda_3)$と固有ベクトル$v_1,v_2,v_3$を求めよ.ただし,固有ベクトルを並べた行列$V = \left(v_1 \hspace{1mm} v_2 \hspace{1mm} v_3\right)$が直交行列となるように固有ベクトルを選べ.
\end{screen}

$x^TAx = (A^Tx)^Tx = x^T(A^Tx) = x^TA^Tx$が成り立つので,$\ds \tilde{A}=\frac{1}{2}\left(A+A^T\right)$と定めればよい.

$\ds \tilde{A} = \begin{pmatrix} -3 & -1 & 1 \\-1 & -3 & -1 \\1 & -1 & -3\end{pmatrix}$
である.特性方程式を解くことにより,固有値$\lambda_1=-1$に対応する固有ベクトル$u_1=\begin{pmatrix}1 \\ -1 \\ 1\end{pmatrix}$,固有値$\lambda_2=\lambda_3 = -4$に対応する固有ベクトル$u_2=\begin{pmatrix} -1 \\ 0 \\ 1\end{pmatrix},u_3=\begin{pmatrix} 1\\ 1 \\ 0\end{pmatrix}$ を得る.

$u_1$を正規化して$\ds v_1 = \frac{1}{\sqrt{3}}\begin{pmatrix}1 \\ -1 \\ 1\end{pmatrix}$,$u_2,u_3$を正規直交化して,$\ds v_2=\frac{1}{\sqrt{2}}\begin{pmatrix} -1 \\ 0 \\ 1\end{pmatrix},v_3=\frac{1}{\sqrt{6}}\begin{pmatrix} 1 \\ 2 \\ 1\end{pmatrix}$ とすれば$V$は直交行列になる.

\begin{screen}
 (5) 任意の実ベクトル$x$に対して,$x^TAx \leq 0$が成り立つことを証明せよ.
\end{screen}

$v_1,v_2,v_3$は(正規直交)基底をなすので,任意の実ベクトル$x$は$x=a_1v_1+a_2v_2+a_3v_3$ と表せる$(a_1,a_2,a_3\in \mathbb{R})$.したがって,$x^TAx = a_1^2\lambda_1+a_2^2\lambda_2+a_3^2\lambda_3 \leq 0$

\begin{screen}
 (6) 関数$g(x)=x^TAx + 2b^Tx$の停留点を求めよ.
\end{screen}

$\ds \pdiff{}{}{x}g(x) = 2\tilde{A}x + 2b = 0$より,$\tilde{A}x=-b$を解いて,$\ds  x=\frac{1}{4}\begin{pmatrix} 1\\ 1 \\0 \end{pmatrix}$

\section{}

\begin{screen}
 $xy$ における曲線に関する以下の設問に答えよ.
\end{screen}
\begin{screen}
 (1)楕円:
 \begin{align*}
  \frac{x^2}{a^2}+\frac{y^2}{b^2}=1 \hspace{2mm} (a>b>0) \hspace{10mm} (\ast)
 \end{align*}
 と双曲線:
 \begin{align*}
  \frac{x^2}{c^2}-\frac{y^2}{d^2}=1 \hspace{2mm} (c>d>0) \hspace{10mm} (\ast\ast)
 \end{align*}
 の焦点の座標が,それぞれ$\ds \left(\pm\sqrt{a^2-b^2},0\right)$ と $\ds\left(\pm\sqrt{c^2+d^2},0\right)$となることを示せ.ここで,楕円は焦点からの距離の和が一定,双曲線は焦点からの距離の差が一定となる曲線である.
\end{screen}


\begin{screen}
 (2) 式$(\ast)$において,$a^2-b^2=u^2$($u$は正の定数)を満たす楕円の集合$E_u$を考える.式$(\ast)$と式$(\ast)$を$x$で微分した微分方程式を連立させ,$E_u$に属する任意の楕円が次の微分方程式を満たすことを示せ.
 \begin{align*}
  xy{y'}^2 + \left(x^2-y^2-u^2\right)y' - xy = 0 \hspace{10mm} (\ast\ast\ast)
 \end{align*}
 ここで,$\ds y' = \diff{}{y}{x}$である.
\end{screen}

\begin{screen}
 (3) 式$(\ast\ast)$において,$c^2+d^2=u^2$を満たす双曲線の集合$H_u$を考える.$H_u$に属する任意の双曲線が式$(\ast\ast\ast)$を満たすことを示せ.
\end{screen}

\begin{screen}
 (4) $E_u$に属するすべての楕円と直交する曲線の集合を$C_u$とする.$C_u$から,直線$x=0$を除き,さらに$y'=0$となる点を含む曲線も除いた集合を$D_u$とする.$D_u$に属する任意の曲線が満たす微分方程式を求めよ.
\end{screen}

\begin{screen}
 (5) 設問(4)で求めた微分方程式を解け.必要であれば,$\ds \alpha=x^2, \beta=y^2, p=\diff{}{\beta}{\alpha}$とおき,$p$に関する微分方程式に書き換えよ.
\end{screen}

\section{}

\begin{screen}
 以下の設問に答えよ.
\end{screen}

\begin{screen}
 (1) $X$ を実数値をとる確率変数とし,$t$を実変数として,$\phi_X(t)$を
 $$\phi_X(t) = E_X\left[e^{tX}\right]$$
 として定める.ここに,$E_X\left[\cdot\right]$は$X$に関する期待値を表す.$\phi_X(t)$が$t=0$の近傍で有限であるとき,$X$の平均と分散を$\phi'_X(0)$と$\phi''_X(0)$を用いて表せ.ここに,$\phi'_X(t)$,$\phi''_X(t)$はそれぞれ$\phi_X(t)$の$t$に関する1階微分,2階微分を表す.
\end{screen}

\begin{screen}
 (2) 互いに独立な確率変数列 $X_1,X_2,\cdots,X_N$ に対して,各 $X_j$ が平均 $\mu$,分散 $\sigma^2$ の同一の1次元正規分布に従うとする.つまり,各$X_j$の従う確率密度関数が,
 $$p(X_j=x) = \frac{1}{\sqrt{2 \pi}\sigma}\exp\left(- \frac{(x-\mu)^2}{2\sigma^2}\right)$$
 で与えられるとする.このとき,$\phi_{X_j}(t)$ を求めよ.また,
 $$ Y = X_1 + X_2 + \cdots + X_N$$
 の従う確率分布を求めよ.ただし,一般に,2つの確率変数$Z$と$W$に対して,$t$の関数として,$\phi_Z(t)=\phi_W(t)$であれば,$Z$と$W$が従う確率分布は等しいという事実を用いてもよい.
\end{screen}

\begin{screen}
 (3) 設問(2)において,$N\in \left\{1,2,\cdots,\infty\right\}$が$\theta(0<\theta<1)$をパラメータとする幾何分布に従って発生するとする.つまり,$N$の確率関数が
 $$P(N=n) = (1-\theta)^{n-1}\theta$$
 で与えられるとする.このとき,$Y=X_1+X_2+\cdots+X_N$に対して,$\phi_Y(t)$を
 $$\phi_Y(t) = E_Y\left[e^{tY}\right]$$
 と定めるとき,$\phi_Y(t)$を求め,それを$\phi_{X_j}(t)$を用いて表せ.ただし,$\phi_{X_j}(t)$は$j$に依らないので,$\phi_X(t)$と書いてよい.
\end{screen}

\begin{screen}
 (4) 設問(3)の$Y$の平均と分散を求めよ.
\end{screen}

\begin{screen}
 (5) $\xi(>E_Y\left[Y\right])$が与えられたとして,設問(3)の$Y$の値が$\xi$を超える確率の上界の1つを$\mu,\sigma,\theta,\xi$の関数として与えよ(必ずしも$\mu,\sigma,\theta,\xi$全てを使わなくてもよい).
\end{screen}

\end{document}
