\documentclass[a4j]{jarticle}
%%  packages
\usepackage{amsmath,amssymb,ascmac}
\usepackage{bm}
\usepackage[dvipdfmx]{graphicx}
\usepackage{listings}
\usepackage[english]{babel}
\lstset{
 	%枠外に行った時の自動改行
 	breaklines = true,
 	%標準の書体
        basicstyle=\ttfamily\footnotesize,
        commentstyle=\footnotesize\bfseries,
        keywordstyle=\footnotesize\bfseries,
 	%枠 "t"は上に線を記載, "T"は上に二重線を記載
	%他オプション:leftline,topline,bottomline,lines,single,shadowbox
 	frame = single,
 	%frameまでの間隔(行番号とプログラムの間)
 	framesep = 5pt,
 	%行番号の位置
 	numbers = left,
	%行番号の間隔
 	stepnumber = 1,
	%タブの大きさ
 	tabsize = 4,
 	%キャプションの場所("tb"ならば上下両方に記載)
 	captionpos = t
}

%% math commands
\let \ds \displaystyle
\newcommand{\idiff}[3]{
  \frac{d^{#1} #2}{d #3^{#1}}
}
\newcommand{\diff}[3]{
  \frac{\mathrm{d}^{#1} #2}{\mathrm{d} #3^{#1}}
}
\newcommand{\pdiff}[3]{
  \frac{\partial^{#1} #2}{\partial #3^{#1}}
}
\newcommand{\intd}[1]{
  \mathrm{d} #1
}


%% title configuration
\title{東京大学大学院情報理工学系研究科2011年度過去問}
\author{}
\date{}


%% headings
\pagestyle{headings}
\markright{東京大学大学院情報理工学系研究科2011年度過去問}




\begin{document}
%%  begin title page
\thispagestyle{empty}
\maketitle
\pagebreak

\section{}

\begin{screen}
 行列$A$ , 行列$B$ , 関数$f(n)$ を以下のように定義する.
 
 $\ds A=\begin{pmatrix}
         a & b & b \\
         b & a & b \\
         b & b & a \\
        \end{pmatrix}$

 $\ds B=\begin{pmatrix}
         0&0&0&a&b&b\\
         0&0&0&b&a&b\\
         0&0&0&b&b&a\\         
         a&b&b&0&0&0\\
         b&a&b&0&0&0\\
         b&b&a&0&0&0\\         
        \end{pmatrix}$
 
 $\ds f(n)=\begin{pmatrix} 1&0&0&0&0&0 \end{pmatrix}B^n
 \left(\begin{array}{c}1\\0\\0\\0\\0\\0\end{array}\right)$

 ただし,$a,b$は$a>0,b>0,a \neq b$なる実数,$n$は正の整数とする.以下の各問に答えよ.
\end{screen}

\begin{screen}
 (1) 行列$A$のすべての固有値を求めよ.また行列$A$の線形独立な3つの固有ベクトルをあげよ.
\end{screen}

\begin{align*}
 \left|\lambda I - A \right| &=
 \begin{vmatrix}
  \lambda -a & -b & -b \\
  -b & \lambda -a & -b \\
  -b & -b & \lambda -a \\
 \end{vmatrix}\\
 &=
 \begin{vmatrix}
  \lambda -a & -b & -b \\
  0 & \lambda -a+b & -(\lambda-a+b) \\
  -b & -b & \lambda -a \\
 \end{vmatrix}\\
 &=
 \begin{vmatrix}
  \lambda -a+b & 0 & -(\lambda-a+b) \\
  0 & \lambda -a+b & -(\lambda-a+b) \\
  -b & -b & \lambda -a \\
 \end{vmatrix}\\
 &=
 \begin{vmatrix}
  \lambda -a+b & 0 & 0 \\
  0 & \lambda -a+b & -(\lambda-a+b) \\
  -b & -b & \lambda -a-b \\
 \end{vmatrix}\\
 &= (\lambda - a + b)^2(\lambda - a -2b) = 0
\end{align*}

より,固有値$\lambda = a-b,a-b,a+2b$.

\begin{itemize}
 \item[(i)] $\lambda=a-b$のとき,固有ベクトル$\left(\begin{array}{c}-2\\1\\1\end{array}\right)$と$\left(\begin{array}{c}0\\-1\\1\end{array}\right)$
 \item[(ii)] $\lambda=a+2b$のとき,固有ベクトル$\left(\begin{array}{c}1\\1\\1\end{array}\right)$
\end{itemize}

\begin{screen}
 (2) 行列$B$のすべての固有値を求めよ.また行列$B$の線形独立な6つの固有ベクトルをあげよ.
\end{screen}

$\ds B =
\left(
\begin{array}{c|c}
 O&A \\
 \hline
 A&O \\
\end{array}
\right),
\bm{x} =
\left(
\begin{array}{c}
 \bm{x}_1\\ \hline 
 \bm{x}_2
\end{array}
\right)$ と表すと,$B\bm{x}=\lambda \bm{x} \Leftrightarrow
\begin{cases}
 A \bm{x}_1 = \lambda \bm{x}_2 \\
 A \bm{x}_2 = \lambda \bm{x}_1 \\
\end{cases}
\Leftrightarrow
\begin{cases}
 A (\bm{x}_1+\bm{x}_2) = \lambda(\bm{x}_1+\bm{x}_2) \\
 A (\bm{x}_1-\bm{x}_2) = -\lambda(\bm{x}_1-\bm{x}_2) \\
\end{cases}
$ 

$\bm{x}_2 \neq \pm \bm{x}_1 $とすると$\pm \lambda$が$A$の固有値となり,$A$の固有値がすべて正であることと矛盾する.

$\bm{x}_2 = \bm{x}_1$のとき,$A \bm{x}_1 = \lambda \bm{x}_1$が必要十分で$\lambda$は$A$の固有値となり,$\bm{x}_1$は固有値$\lambda$に対応する固有ベクトルである.

よって,固有値$a-b$に対応する固有ベクトル$\left(\begin{array}{c}-2\\1\\1\\-2\\1\\1\end{array}\right)$と$\left(\begin{array}{c}0\\-1\\1\\0\\-1\\1\end{array}\right)$

固有値$a+2b$に対応する固有ベクトル$\left(\begin{array}{c}1\\1\\1\\1\\1\\1\end{array}\right)$

$\bm{x}_2 = -\bm{x}_1$のとき,$A \bm{x}_1 = -\lambda \bm{x}_1$が必要十分で$-\lambda$は$A$の固有値となり,$\bm{x}_1$は固有値$-\lambda$に対応する固有ベクトルである.

よって,固有値$-(a-b)$に対応する固有ベクトル$\left(\begin{array}{c}-2\\1\\1\\2\\-1\\-1\end{array}\right)$と$\left(\begin{array}{c}0\\-1\\1\\0\\1\\-1\end{array}\right)$

固有値$-(a+2b)$に対応する固有ベクトル$\left(\begin{array}{c}1\\1\\1\\-1\\-1\\-1\end{array}\right)$

\begin{screen}
 (3) $f(1),f(2),f(3)$ を求めよ.
\end{screen}

$f(1)=0,f(2)=a^2+2b^2,f(3)=0$

\begin{screen}
 (4) $a=3,b=2$のとき$f(n)$を求めよ.
\end{screen}

(略解)$B$は対称行列であり,正規直交化された固有ベクトルを並べた直交行列$V$により対角化される.
$\ds f(n)=\begin{pmatrix} 1&0&0&0&0&0 \end{pmatrix}V
\left(
\begin{array}{cccccc}
 1 & 0 & 0 & 0 & 0 & 0 \\
 0 & 1 & 0 & 0 & 0 & 0 \\
 0 & 0 & 7 & 0 & 0 & 0 \\
 0 & 0 & 0 &-1 & 0 & 0 \\
 0 & 0 & 0 & 0 &-1 & 0 \\
 0 & 0 & 0 & 0 & 0 &-7 \\
\end{array}
\right)^n V^\top\left(\begin{array}{c}1\\0\\0\\0\\0\\0\end{array}\right)$によって直接一般項が求まる.

\end{document}
