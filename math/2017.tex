\documentclass[a4j]{jarticle}
%%  packages
\usepackage{amsmath,amssymb,ascmac}
\usepackage{bm}
\usepackage[dvipdfmx]{graphicx}
\usepackage{listings}
\usepackage[english]{babel}
\lstset{
 	%枠外に行った時の自動改行
 	breaklines = true,
 	%標準の書体
        basicstyle=\ttfamily\footnotesize,
        commentstyle=\footnotesize\bfseries,
        keywordstyle=\footnotesize\bfseries,
 	%枠 "t"は上に線を記載, "T"は上に二重線を記載
	%他オプション:leftline,topline,bottomline,lines,single,shadowbox
 	frame = single,
 	%frameまでの間隔(行番号とプログラムの間)
 	framesep = 5pt,
 	%行番号の位置
 	numbers = left,
	%行番号の間隔
 	stepnumber = 1,
	%タブの大きさ
 	tabsize = 4,
 	%キャプションの場所("tb"ならば上下両方に記載)
 	captionpos = t
}

%% math commands
\let \ds \displaystyle
\newcommand{\idiff}[3]{
  \frac{d^{#1} #2}{d #3^{#1}}
}
\newcommand{\diff}[3]{
  \frac{\mathrm{d}^{#1} #2}{\mathrm{d} #3^{#1}}
}
\newcommand{\pdiff}[3]{
  \frac{\partial^{#1} #2}{\partial #3^{#1}}
}



%% title configuration
\title{東京大学大学院情報理工学系研究科2017年度過去問}
\author{}
\date{}


%% headings
\pagestyle{headings}
\markright{東京大学大学院情報理工学系研究科2017年度過去問}




\begin{document}
%%  begin title page
\thispagestyle{empty}
\maketitle
\pagebreak

\section{}

\begin{screen}
 3次元ベクトル $\begin{pmatrix} x_n \\ y_n \\ z_n \end{pmatrix}$は式
 $$\begin{pmatrix} x_{n+1} \\ y_{n+1} \\ z_{n+1} \end{pmatrix} = A\begin{pmatrix} x_n \\ y_n \\ z_n \end{pmatrix} \quad\quad (n=0,1,2,\cdots)$$
 を満たすものとする.ただし,$x_0,y_0,z_0,\alpha$は実数とし,
 $$A=
 \begin{pmatrix}
  1 - 2\alpha & \alpha & \alpha \\
  \alpha & 1-\alpha & 0 \\
  \alpha &  0 & 1-\alpha
 \end{pmatrix}, \quad\quad 0<\alpha < \frac{1}{3}$$
 とする.以下の問いに答えよ.
\end{screen}


\begin{screen}
 (1) $x_n+y_n+z_n$ を $x_0,y_0,z_0$ を用いて表せ.
\end{screen}


\begin{screen}
 (2) 行列 $A$ の固有値 $\lambda_1,\lambda_2,\lambda_3$ と,それぞれの固有値に対応する固有ベクトル $\bm{v}_1,\bm{v}_2,\bm{v}_3$ を求めよ.
\end{screen}


\begin{screen}
 (3) 行列 $A$ を $\lambda_1,\lambda_2,\lambda_3,\bm{v}_1,\bm{v}_2,\bm{v}_3$ を用いて表せ.
\end{screen}


\begin{screen}
 (4) $\begin{pmatrix} x_n \\ y_n \\ z_n \end{pmatrix}$ を$x_0,y_0,z_0,\alpha$を用いて表せ.
\end{screen}

\begin{screen}
 (5) $\ds \lim_{n\rightarrow\infty}\begin{pmatrix} x_n \\ y_n \\ z_n \end{pmatrix}$を求めよ.
\end{screen}

\begin{screen}
 (6) 以下の式
 $$f(x_0,y_0,z_0)=\frac{(x_n,y_n,z_n)\begin{pmatrix} x_{n+1} \\ y_{n+1} \\ z_{n+1} \end{pmatrix}}{(x_n,y_n,z_n)\begin{pmatrix} x_n \\ y_n \\ z_n \end{pmatrix}}$$
 を$x_0,y_0,z_0$の関数とみなして, $f(x_0,y_0,z_0)$の最大値および最小値を求めよ.ただし,\\ $x_0^2+y_0^2+z_0^2 \neq 0$ とする.
\end{screen}

\section{}

\begin{screen}
 実数値関数 $u(x,t)$ が $0 \leq 1,$ $t\geq 0$ で定義されている.ここで,$x$と$t$は互いに独立である.偏微分方程式
\begin{align*}
 \pdiff{}{u}{t} = \pdiff{2}{u}{x} \tag{$\ast$} \\
\end{align*}
 の解を次の条件
 \begin{align*}
  \mbox{境界条件:} & u(0,t) = u(1,t) = 0 \\
  \mbox{初期条件:} & u(x,0) = x-x^2\\
 \end{align*}
 のもとで求める.ただし,定数関数$u(x,t)=0$は明らかに解であるから,それ以外の解を考える.以下の問いに答えよ.
\end{screen}


\begin{screen}
 (1) 次の式を計算せよ.ここで,$n,$ $m$ はともに正の整数とする.
 $$\int_0^1 \sin(n \pi x) \sin(m \pi x) \mathrm{d}x$$
\end{screen}

$\ds I = \int_0^1 \sin(n \pi x) \sin(m \pi x) \mathrm{d}x$ とおく.
\begin{align*}
 I &= \int_0^1 \sin(n \pi x) \sin(m \pi x) \mathrm{d}x \\
 &= \left[- \frac{1}{m \pi} \sin(n \pi x) \cos(m \pi x)\right]_0^1 + \frac{n}{m} \int_0^1 \cos(n \pi x) \cos(m \pi x) \mathrm{d} x \\
 &= \left[\frac{n}{m^2 \pi } \cos(n \pi x)\sin(m \pi x)\right]_0^1 + \frac{n^2}{m^2} \int_0^1 \sin(n \pi x) \sin(m \pi x) \mathrm{d}x \\
 &= \frac{n^2}{m^2} I 
\end{align*}

\begin{itemize}
 \item[(i)] $m \neq n$ の場合.$I=0$
 \item[(ii)] $m = n$ の場合.
             \begin{align*}
              I &= \int_0^1 \sin^2 (n \pi x) \mathrm{d}x \\
              &= \frac{1}{2} \int_0^1 \Bigl(1-\cos(2n\pi x)\Bigr) \mathrm{d}x \\
              \therefore I&= \frac{1}{2}
             \end{align*}
\end{itemize}

\begin{screen}
 (2) $x$のみの関数$\xi(x)$ および $t$のみの関数$\tau(t)$ を用いて,$u(x,t)=\xi(x)\tau(t)$ とおけるとする.任意の定数$C$を用いて,$\xi$および$\tau$が満たす常微分方程式をそれぞれ表せ.関数$f(x)$と関数$g(t)$が任意の$x$と$t$について$f(x)=g(t)$を満たす場合は,$f(x)$と$g(t)$が定数関数となることを用いてもよい.
\end{screen}

$(\ast)$に$u(x,t)=\xi(x)\tau(t)$を代入すると,
\begin{align*}
 \xi(x) \cdot \pdiff{}{\tau}{t}(t) &= \pdiff{2}{\xi}{x}(x) \cdot \tau(t) \\
 \intertext{変数分離して,題で与えられた事実を用いると}
 \frac{\ds\pdiff{}{\tau}{t}(t)}{\tau(t)} &= \frac{\ds\pdiff{2}{\xi}{x}(x)}{\xi(x)} = C
\end{align*}

よって満たすべき常微分方程式は
$$
\begin{cases}
 \ds\idiff{}{\tau}{t} = C\tau(t) \\
 \\
 \ds\idiff{2}{\xi}{x} = C\xi(x)
\end{cases}
$$

\begin{screen}
 (3) 設問(2) の常微分方程式を解け.次に,境界条件を満たす偏微分方程式$(\ast)$の解の一つが次の式で表される $u_n(x,t)$ で与えられることを示し,$\alpha,$ $\beta$ を正の整数 $n$を用いて表せ.
 $$u_n(x,t) = e^{\alpha t}\sin(\beta x)$$
\end{screen}

(2)の常微分方程式を解くと,$\xi(x) = Ae^{\sqrt{C}x}+Be^{-\sqrt{C}x},$ $\tau(t) = De^{Ct}$ となる.境界条件から$\xi(0)=\xi(1)=0 \Leftrightarrow A+B=0 \land Ae^{\sqrt{C}}+Be^{-\sqrt{C}} = 0 \Rightarrow e^{\sqrt{C}}=e^{-\sqrt{C}} \Rightarrow C=-n^2\pi^2$

$A,B,D$を適当に定めれば $u_n(x,t) = e^{-n^2\pi^2 t} \sin(n \pi x)$ つまり,$\alpha = -n^2\pi^2, \beta = n \pi$となる.

\begin{screen}
 (4) 境界条件と初期条件を満たす偏微分方程式$(\ast)$の解は $u_n(x,t)$ の線型結合として次の式で表される.$c_n$ を求めよ.設問(1)の結果を用いてもよい.
 $$u(x,t) = \sum_{n=1}^\infty c_n u_n (x,t)$$
\end{screen}

初期条件$u_0 (x) = x-x^2$ とおく.$m$を正の整数として

\begin{align*}
 \int_0^1 \sin(m \pi x)u_0(x) \mathrm{d} x&=  \int_0^1 \sin(m \pi x)u(x,0) \mathrm{d} x \\
 \int_0^1 (x-x^2)\sin(m \pi x) \mathrm{d} x&=  \int_0^1 \sin(m \pi x)\sum_{n=1}^\infty c_n u_n(x,0) \mathrm{d} x \\
 \frac{1}{m\pi} \int_0^1(1-2x)\cos(m \pi x) \mathrm{d}x &= \sum_{n=1}^\infty c_n \int_0^1 \sin(n \pi x) \sin( m \pi x) \mathrm{d} x \\
 \frac{2}{m^2\pi^2}\int_0^1 \sin(m \pi x)\mathrm{d}x&= \frac{c_m}{2}
\end{align*}

$$ \therefore c_n = \frac{4}{n^2\pi^2}\int_0^1 \sin(n \pi x)\mathrm{d}x = \frac{4}{n^3\pi^3}(1-\cos(n \pi))$$

\end{document}
